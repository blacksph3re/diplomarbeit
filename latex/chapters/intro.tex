\label{ch:intro}

To tackle the climate crisis, renewable energies play a vital role. Wind power is one of the main technologies driving the transition to renewables. While the political effort behind this transition fluctuates, continuously improving economic competetiveness is a major driving factor behind the steadily growing capacity of wind energy. Competetiveness of an energy source is mainly determined by a metric called \acf{LCOE}, which is defined as the total energy production divided by the total cost of running the energy source \cite{kostLevelizedCostElectricity2021}. Any reduction of \ac{LCOE} directly contributes to further adoption of wind energy. Lowering $LCOE = \frac{costs}{energy}$ always involves either increasing energy yield without increasing costs, decreasing costs without decreasing energy or a combination of both. Increasing energy yield for a fixed size turbine is achieved mainly through optimizing the blade airfoil and reducing power losses in the drive train and generator. However, the optimization margin on that end has become relatively slim. Modern wind turbines exhibit generator losses under 5\%, and aerodynamic rotor efficiency of over 49\% at rated speed, close to the theoretical maximum of 59.3\% \cite{bortolottiIEAWindTCP2019} \cite{raghebWindTurbinesTheory2011}. 

This work focusses on reducing costs without impacting energy yield. For offshore installations, the top three cost factors for wind turbine operation are maintenance (34.3\%), turbine capital expenditures (21\%) and construction of the substructure (13.2\%) \cite[Figure ES2]{stehly2019CostWind2020}, making up approximately two thirds of the total costs. These three top cost factors are directly influenced by the loads a turbine experiences throughout its lifetime. Fatigue loads are a major factor behind component wear, which drives maintenance costs. Expected fatigue and extreme loads are design driving factors for both the turbine and substructure, as the strength and thus cost of the materials is dictated by the forces they have to withstand. Reducing extreme and fatigue loads directly lowers turbine cost with a significant impact on \ac{LCOE}.

A wind turbine controller has the task to maximize energy yield while keeping the load level of the turbine at an acceptable threshold to not endanger the turbine integrity. Especially at higher wind speeds, a good load control policy has the potential to alleviate loads with minimal impact on energy yields. With their \ac{IPC} strategy, \citet{bossanyiIndividualBladePitch2003} have developed such a load minimizing controller for wind turbines, and the concept has seen several iterations. Solutions based on PID controllers \cite{bossanyiFurtherLoadReductions2005}, \ac{LQR} \cite{bossanyiIndividualBladePitch2003} and \ac{MPC} \cite{petrovicMPCFrameworkConstrained2021} are the most prominent. While each has its advantages and disadvantages, these are all linear control methods and as such fundamentally similar \cite{lioFundamentalPerformanceSimilarities2017}. Linear controllers fail to address the non-linear and stochastic aspects of wind turbine control, which play a bigger role especially on large modern turbines. Furthermore, integrating high-dimensional inputs like LIDAR measurements or controlling a high number of outputs like active flaps on the blade is in principle promising, but difficult to achieve with traditional control policies \cite{bossanyiWindTurbineControl2014} \cite{perez-beckerActiveFlapControl2021} \cite{barlasReviewStateArt2010}.

\acf{RL} has proven to be capable of high-dimensional non-linear optimization. Integrating high dimensional data like pixel output of computer games in \citet{mnihPlayingAtariDeep2013} is common in the field. Working with high dimensional output like joint torques for complex robots is a common benchmark in \ac{RL} \cite{brockmanOpenAIGym2016}. Also, neural networks have shown great potential to model complicated relationships of highly uncertain systems, including wind phenomena \cite{demolliWindPowerForecasting2019}. All this combined, reinforcement learning is a promising candidate for optimizing wind turbine control. However, previous works have not shown great success trying to achieve wind turbine load control by reinforcement learning \cite{coqueletBiomimeticIndividualPitch2020}.

This work introduces \acf{WINDL}. We explore the use of \ac{RL} as a possible candidate for specialized wind turbine load control with a focus on blade load and pitch load reduction. This proof of concept aims to serve as a starting point for future controller developments, addressing the fundamental limitations of linear control and the specific challenges in wind turbine load control. We compare our solution to two state-of-the-art control strategies quantitatively and qualitatively to understand its strengths and weaknesses. Furthermore, we analyze the effect of hyperparameters, the training process and the inner workings of the \ac{RL} algorithm to facilitate adoption and improvements of \ac{WINDL}.


\section*{Structure Of The Work}

\begin{summary}
To allow for a fast read, the first paragraph in every section summarizes the key concepts and findings of that section in a few works. It is indented as this paragraph.
\end{summary}

We start this work with a background chapter (Chapter \ref{ch:background}), introducing fundamentals from reinforcement learning and wind turbine control. This chapter aims to introduce the notation and to aid researchers in related fields to understand our work. In the reinforcement learning background section (Section \ref{section:background-reinforcement-learning}), we introduce fundamentals of reinforcement learning and present the setup of actor-critic methods. Most crucially, this section introduces the specific algorithm in use for most of our work. For an in-depth introduction to the field of \ac{RL}, we recommend the book by \citet{suttonReinforcementLearningIntroduction2018}. Similarly, the wind turbine control section (Section \ref{section:background-wind-turbine-control}) only introduces the most relevant concepts of wind turbine control to this work. Among them is a general overview of wind turbine control and common techniques for wind turbine pitch control. For a more in-depth introduction to wind turbines, we recommend the book by \citet{burtonWindEnergyHandbook2011}.

In the approach chapter (Chapter \ref{ch:approach}), we present our approach \ac{WINDL}. The chapter is constituted of two parts, a section introducing \ac{WINDL} formally (Section \ref{section:approach-theory}) and a section presenting implementation details (Section \ref{section:approach-implementation}). The formal section should enable anyone to replicate our work or build upon it. It gives an abstract understanding of the challenges and solutions involved in applying reinforcement learning to wind turbine load control. The second section with implementation details presents technical problems which are specific to the chosen reinforcement learning implementation, wind turbine simulation and other tools used in the process. This section might not apply to a different set of tools and implementations.

Following the approach chapter, we present our evaluation results (Chapter \ref{ch:results}). We structure this chapter based on claims and limitations, which we investigate individually. For each claim, we present evidence showing the potential of the approach, and how it is superior and inferior to existing baseline wind turbine controllers. This is achieved by comparing against two baseline control strategies, which are common to modern (\ac{IPC}) and older (\ac{CPC}) real-world turbines.

In Chapter \ref{ch:related-work}, we compare our work to related works. A special focus is put on works that directly compete with our solution, but we give an overview of more loosely related work as well. Comparing the advantages and disadvantages of related solutions, we outline the specific niche which could be filled by \ac{WINDL} controllers.

Finally, we conclude our results and present opportunities for future work in Chapter \ref{ch:conclusion}.

